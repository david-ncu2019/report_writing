\documentclass[12pt, a4paper]{article}
\usepackage[left=2.5cm, right=2.5cm, top=2.5cm, bottom=2.5cm]{geometry}
\usepackage{amsmath}
\usepackage{graphicx}
\usepackage{hyperref}
\usepackage{times}
\usepackage{fancyhdr}

% Sets up a simple header for the document
\pagestyle{fancy}
\fancyhf{}
\rhead{Progress Report}
\lhead{2024-2025}
\cfoot{\thepage}

\title{
	\huge{\textbf{Academic and Research Progress Report (2024-2025)}} \\
	\vspace{1cm}
	\large{Submitted to the Scholarship Committee} \\
	\vspace{2cm}
}
\author{\textit{Your Name}}
\date{\today}

\begin{document}
	
	\maketitle
	\thispagestyle{empty}
	\newpage
	
	\tableofcontents
	\newpage
	
	\section{Introduction and Overview}
	
	This report details my academic and research progress throughout the 2024-2025 academic year. This period has been marked by significant scholarly development, highlighted by the publication of my first peer-reviewed paper, active engagement with the international scientific community through multiple conference presentations, and substantial advancement on a second manuscript. My research continues to address the critical environmental challenge of land subsidence by integrating advanced remote sensing with hydrogeological modeling to develop practical mitigation strategies.
	
	This document aims to demonstrate my continued dedication and the substantive progress made possible through the support of this scholarship. Furthermore, it outlines my key accomplishments and establishes the future direction of my work, illustrating how my research contributes to both scientific understanding and practical solutions for sustainable water management and infrastructure protection.
	
	\section{Summary of Main Activities (2024-2025)}
	
	\subsection{Conference Participation and Scientific Community Engagement}
	
	During this academic year, I actively participated in three major international conferences, demonstrating my commitment to scientific dissemination and community engagement. In July 2024, I attended the 21st Annual Meeting of the Asia Oceania Geosciences Society (AOGS) in Pyeongchang, South Korea, where I presented two oral presentations. The first presentation, titled "Monitoring and Assessing Environmental Risks of Land Subsidence in the Choushui River Fluvial Plain by Multiple-Source Data Integration," was delivered in the session IG06 - Near Surface Investigation and Modeling for Groundwater Resources Assessment. This presentation integrated data from multiple monitoring systems including groundwater level monitoring wells, GPS stations, survey leveling benchmarks, multi-level compaction monitoring wells, and InSAR to investigate subsidence patterns. The second presentation focused on "Experimental and Numerical Assessment of Fresh Water and Seawater Interactions in the Coastal Aquifer of the Taoyuan Tableland, Taiwan," which demonstrated my broader expertise in hydrogeological modeling and experimental techniques.
	
	Following the success of my AOGS participation, I presented at the 2025 Taiwan Geosciences Assembly (TGA) in Taipei during June 2025. This presentation outlined the conceptual framework for my second manuscript, proposing an approach that integrates SBAS-InSAR with multilayer compaction monitoring well data to reveal regional patterns of layerwise compaction through geographically weighted regression. Most recently, from August 3-8, 2025, I attended the IEEE International Geoscience and Remote Sensing Symposium (IGARSS) in Brisbane, Australia, where I presented a poster showcasing the findings from my published work on land subsidence monitoring and mitigation strategies.
	
	\subsection{First-Author Publication Achievement}
	
	My first-author publication in the peer-reviewed journal Remote Sensing presented a comprehensive study on land subsidence in Taiwan's critical Choushui River Fluvial Plain. The core of this work was the development of a robust framework that integrates dense, time-series InSAR measurements (from 292 Sentinel-1 images) with sparse, site-specific hydrogeological observations. Our analysis identified a major subsidence bowl with sinking rates up to 60 mm/year, and quantitatively demonstrated its strong correlation with both fine-grained sedimentary materials and groundwater level decline. Crucially, this research provided two actionable outcomes: a quantitative risk assessment for the Taiwan High-Speed Rail system traversing the region, and a predictive model that establishes clear management thresholds for sustainable groundwater extraction.
	
%====================================================================
% VERSION TINH CHỈNH BỞI "ÔNG ANH"
% Mục tiêu: Gọn hơn, logic hơn, và đặt GTWR vào đúng vai trò trung tâm.
%====================================================================

\section{Second Manuscript: A Spatiotemporal Data Fusion Framework for High-Resolution Subsurface Compaction Mapping}
% === BÌNH LUẬN CỦA ANH ===
% Cái tiêu đề này nó cụ thể và "hot" hơn. Nó nêu bật được 2 từ khóa: 
% "Data Fusion Framework" (khung làm việc dung hợp dữ liệu) và 
% "High-Resolution Subsurface Compaction Mapping" (thành quả cuối cùng).
% Nó "bán" được sản phẩm của em ngay từ tiêu đề.

\subsection{Research Motivation: From Surface Phenomena to Subsurface Mechanisms}
%
% === BÌNH LUẬN CỦA ANH ===
% Đoạn này gộp 2 phần Conceptual Framework và Research Motivation lại.
% Mục tiêu là trình bày vấn đề một cách súc tích.
%
My first-author publication (\textit{Nguyen et al., 2024}) established a robust methodology for quantifying large-scale surface deformation in the Choushui River Fluvial Plain using time-series InSAR. While this provided a critical overview of the subsidence phenomenon, it highlighted two fundamental research gaps that motivate my current work: 
%
\begin{enumerate}
	\item \textbf{The Scale Mismatch Problem:} There is a significant disconnect between dense, surface-level InSAR data and sparse, depth-resolved compaction data from MLCWs. How can we reliably estimate subsurface processes at locations without expensive, direct instrumentation?
	\item \textbf{The Linearity Assumption Problem:} The relationship between groundwater dynamics and compaction is inherently non-linear and varies significantly across different geological settings (spatial non-stationarity) and seasons (temporal non-stationarity). Simple linear models fail to capture this complexity.
\end{enumerate}
%
This manuscript addresses these gaps by proposing an innovative framework to create the first high-resolution, spatiotemporal maps of layer-wise compaction for the entire region.

\subsection{Methodological Framework: Spatiotemporal Data Fusion via GTWR}
%
% === BÌNH LUẬN CỦA ANH ===
% Đây là phần thay đổi lớn nhất. Anh đã gộp cả 3 "thành phần" của em lại
% và đặt GTWR làm giải pháp trung tâm, thay vì xem nó như một trong ba thứ.
%
To bridge the gap between surface and subsurface data, our core methodological innovation is the application and validation of a \textbf{Geographically and Temporally Weighted Regression (GTWR)} framework. This approach is uniquely suited to this problem because it is explicitly designed to model non-stationary relationships that vary in both space and time.

The specific implementation is formulated as:
%
\begin{equation}
	\text{MLCW}_{i,n}(t) = \beta_{0}(u_i, v_i, t) + \beta_{1}(u_i, v_i, t) \cdot \text{InSAR}_{i}(t) + \varepsilon_i
\end{equation}
%
where the layer-wise compaction in layer \textit{n} at location \textit{i} and time \textit{t} (MLCW$_{i,n}(t)$) is modeled as a local, time-varying function of the surface deformation (InSAR$_{i}(t)$). The spatiotemporally varying coefficients, $\beta_{0}$ (local intercept) and $\beta_{1}$ (dynamic sensitivity), capture the underlying hydrogeological mechanisms. 

To ensure the robustness of our predictor variable, the InSAR time-series is derived from the synergistic fusion of both **ascending and descending Sentinel-1 orbits**, allowing us to decompose the satellite's line-of-sight measurements into vertical and East-West horizontal motion components. High-precision GPS data are then used to calibrate the resulting deformation field, creating a reliable input for the GTWR model. The calibration of the GTWR model itself is performed systematically for each aquifer layer, optimizing the spatiotemporal bandwidth and parameters using a leave-one-station-out cross-validation scheme to prevent overfitting and assess the model's predictive power.

\subsection{Anticipated Contributions and Scientific Insights}
%
% === BÌNH LUẬN CỦA ANH ===
% Đoạn này tóm gọn lại phần "đóng góp" cho nó sắc bén và mạnh mẽ.
%
The successful implementation of this GTWR framework is expected to yield several novel contributions:
%
\begin{itemize}
	\item \textbf{First-of-its-kind Subsurface Maps:} Generation of the first-ever continuous, high-resolution maps of layer-wise compaction across the Choushui River Plain, moving beyond the limitations of sparse point measurements.
	
	\item \textbf{Quantitative Evidence for Hydrogeological Mechanisms:} The resulting spatiotemporal maps of the $\beta$ coefficients will serve as a powerful proxy for subsurface properties. Their spatial patterns will be used to quantitatively validate conceptual models (e.g., the "sponge vs. isolated box" hypothesis), and their temporal variations will provide direct evidence of non-linear processes like inelastic compaction and hysteresis.
	
	\item \textbf{Enhanced Risk Assessment and Management Tools:} The predicted compaction maps, coupled with their associated uncertainty fields, will provide a direct, data-driven tool for assessing risks to critical infrastructure and for guiding future groundwater management policies and monitoring network design.
\end{itemize}
	
	\subsection{Timeline for Completion and Dissemination}
	
	I have established a detailed timeline for completing this research and disseminating the results to maximize impact within the scientific community and practical applications. During the fourth quarter of 2025, I will complete the processing and integration of all remote sensing and ground-based datasets. This phase includes finalizing the three-dimensional deformation maps, completing the machine learning model training and validation, and implementing the geographically weighted regression framework.
	
	The first quarter of 2026 will be dedicated to comprehensive analysis of the results and manuscript preparation. This phase will involve generating all figures and tables, conducting thorough statistical analysis of model performance, and writing the complete manuscript text. The manuscript will undergo internal review by my supervisors and collaborators to ensure scientific rigor and clarity before submission.
	
	Following this preparation phase, I plan to submit the manuscript during the second quarter of 2026 to a premier journal in the field, with \textit{Remote Sensing of Environment} as the primary target due to its emphasis on innovative remote sensing applications and high impact factor within the Earth sciences community. The journal's focus on advanced remote sensing techniques and their integration with ground-based observations aligns perfectly with the methodological contributions of this research.
	
	\section{Conclusion}
	
	This academic year has been exceptionally productive, marked by significant achievements in research publication, scientific community engagement, and methodological advancement. The publication of my first-author paper in \textit{Remote Sensing} established my research contributions to the field of land subsidence monitoring and mitigation. My active participation in three major international conferences has strengthened my professional network and provided valuable feedback that has enhanced my ongoing research.
	
	The development of my second manuscript represents a substantial methodological advancement that addresses the acknowledged limitations of my previous work while pushing the boundaries of what is possible in subsidence monitoring and prediction. The integration of three-dimensional deformation mapping, machine learning approaches, and geographically weighted regression creates a comprehensive framework that will advance both scientific understanding and practical applications in groundwater management and infrastructure protection.
	
	This research trajectory demonstrates the iterative nature of scientific advancement, where each study builds upon previous work while addressing new challenges and expanding methodological capabilities. I am confident in the direction of my research and remain deeply grateful for the support of this scholarship, which has been instrumental in enabling these achievements. The financial and institutional support has allowed me to focus entirely on my research objectives while participating fully in the international scientific community.
	
	As I look toward the completion of my second manuscript, I am excited about its potential to make significant contributions to the field while providing practical tools for addressing one of Taiwan's most pressing environmental challenges. I am committed to maintaining this trajectory of excellence and am grateful for the continued opportunity to pursue these important scientific questions with the support of this prestigious scholarship.
	
\end{document}